%*******************************************************
% Abstract
%*******************************************************
%\renewcommand{\abstractname}{Abstract}
\pdfbookmark[1]{Abstract}{Abstract}
\begingroup
\let\clearpage\relax
\let\cleardoublepage\relax
\let\cleardoublepage\relax

\chapter*{Abstract}

Modeling dynamical systems is a core subject of many scientific disciplines as it allows to predict future states, understand complex interactions over time, and enable informed decision-making.
Biological systems in particular are governed by dynamical processes, with their inherently complex and constantly changing patterns of interactions and behaviors.
Single-cell biology thereby has revolutionized biomedical research, as it allows to monitor such systems at unprecedented scales.
At the same time, it presents us with formidable challenges: While single-cell high-throughput methods routinely produce millions of data points, they are destructive assays, such that the same cell cannot be observed twice nor fully profiled over time.
Since many of the most pressing questions in the field involve modeling the dynamic responses of heterogeneous cell populations to various stimuli, such as developmental signals, genetic perturbations, or drug treatments, there is a pressing need to provide computational methods that can circumvent that limitation and re-align these unpaired snapshots.

Optimal transport (OT) has emerged as a major opportunity to fill in that gap \textit{in silico} as it allows us to reconstruct how a distribution evolves over time, given only access to \emph{distinct snapshots} of \emph{unaligned} data points.
Classical OT methods, however, do not generalize to \emph{unseen} samples. Yet, this is crucial when, for example, predicting treatment responses of incoming patient samples, or extrapolating cellular dynamics beyond the measured horizon.

Harnessing the theoretical constructs of OT, this thesis embarks on the exploration and development of \textbf{static} and \textbf{dynamic neural optimal transport schemes} for elucidating the intricate dynamics of biological populations. It encapsulates an array of algorithmic frameworks, with contributions to both the understanding and prediction of population dynamics:

First, we derive \emph{static} neural optimal transport schemes, capable of mapping and interpreting unpaired distributions of perturbed and unperturbed cells. These models excel at predicting single-cell responses to varying perturbations such as cancer drug screens. It has profound implications for personalized medicine as it opens new frontiers to predict treatment responses of unseen patients in large-scale clinical studies.

Second, we explore several \emph{dynamic} optimal transport formulations that leverage the connections of OT to partial differential equations and gradient flows through the Jordan-Kinderlehrer-Otto (JKO) scheme, as well as stochastic differential equations and optimal control through diffusion Schr{\"o}dinger bridges. These methods then serve as robust tools for reconstructing stochastic and continuous-time dynamics from marginal observations, allowing us to dissect fine-grained and time-resolved cellular mechanisms.

This thesis thereby weaves together a range of seemingly disparate concepts under a unified framework and the methodologies presented provide the computational and mathematical foundation for modeling cellular dynamics, offering new avenues for understanding cellular heterogeneity, plasticity, and response landscapes.
By providing \emph{neural} parameterizations of static and dynamic OT for out-of-sample inference, the developed methods anticipate broad implications, spanning from understanding developmental trajectories to predicting patient-specific drug responses and designing personalized cancer therapies.

% Cell populations are almost always heterogeneous in function and fate. To understand the plasticity of cells and their responses to molecular perturbations, such as drugs or developmental signals, it is vital to recover the underlying population dynamics and fate decisions of single cells. However, measuring features of single cells requires destroying them. As a result, a cell population can only be monitored with sequential snapshots, obtained by sampling a few particles that are sacrificed in exchange for measurements.

% This celebrated theory provides the mathematical link that unifies the several contributions to model cellular dynamics that we present here: 
\endgroup

\cleardoublepage%

\begingroup
\let\clearpage\relax
\let\cleardoublepage\relax
\let\cleardoublepage\relax

\begin{otherlanguage}{ngerman}
\pdfbookmark[1]{Zusammenfassung}{Zusammenfassung}
\chapter*{Zusammenfassung}

Die Modellierung dynamischer Systeme ist ein zentraler Bestandteil vieler wissenschaftlicher Disziplinen, da sie die Vorhersage zuk{\"u}nftiger Zust{\"a}nde, das Verst{\"a}ndnis komplexer Interaktionen {\"u}ber die Zeit und fundierte Entscheidungsfindung erm{\"o}glicht. Insbesondere biologische Systeme werden von dynamischen Prozessen gesteuert, mit ihren inh{\"a}rent komplexen und st{\"a}ndig wechselnden Interaktions- und Verhaltensmustern. Einzelzellbiologie hat dabei die biomedizinische Forschung revolutioniert, da sie es erm{\"o}glicht, solche Systeme in noch nie da gewesenen Ma{\ss}t{\"a}ben zu beobachten. Gleichzeitig stellt sie Biologen vor gewaltige Herausforderungen: W{\"a}hrend Einzelzell-Hochdurchsatzmethoden routinem{\"a}{\ss}ig Millionen von Datenpunkten produzieren, handelt es sich dabei um destruktive Assays, so dass dieselbe Zelle nicht zweimal beobachtet oder vollst{\"a}ndig {\"u}ber die Zeit erfasst werden kann. Da viele der dringendsten Fragen in diesem Feld die Modellierung der dynamischen Reaktionen heterogener Zellpopulationen auf verschiedene Stimuli, wie Entwicklungssignale, genetische St{\"o}rungen oder Medikamentenbehandlungen, betreffen, besteht ein dringender Bedarf an Berechnungsmethoden, die diese Einschr{\"a}nkung umgehen k{\"o}nnen.

Optimaler Transport hat sich als eine gro{\ss}e Chance herausgestellt, diese L{\"u}cke \textit{in silico} zu schlie{\ss}en, da er uns erlaubt, zu rekonstruieren, wie eine Verteilung im Laufe der Zeit ver{\"a}ndert, wenn nur Zugang zu \emph{unterschiedlichen Momentaufnahmen} von \emph{nicht ausgerichteten} Proben gegeben ist. Klassische OT-Methoden verallgemeinern jedoch nicht auf \emph{unbekannte} Proben. Dies ist jedoch entscheidend, um zum Beispiel Reaktionen neu eingehender Patientenzellen vorherzusagen oder zellul{\"a}re Dynamiken {\"u}ber den aufgezeichneten Horizont hinaus zu extrapolieren.

Die pr{\"a}sentierte Arbeit fasst eine Reihe von algorithmischen Rahmenwerken zusammen, mit wegweisenden Beitr{\"a}gen sowohl zum Verst{\"a}ndnis als auch zur Vorhersage von Populationsdynamiken. Unter Ausnutzung der theoretischen Konstrukte von OT begibt sich diese Dissertation auf die Erforschung und Entwicklung von statischen und dynamischen \emph{neuronalen} optimalen Transport-Schemata zur Erl{\"a}uterung der komplexen Dynamik biologischer Populationen.

Zun{\"a}chst leiten wir \emph{statische} neuronale optimale Transport-Schemata ab, die in der Lage sind, ungepaarte Verteilungen von gest{\"o}rten und ungest{\"o}rten Zellen zu erfassen und zu interpretieren. Diese Modelle sind hervorragend in der Vorhersage von Einzelzellreaktionen auf unterschiedliche St{\"o}rungen und in der Ber{\"u}cksichtigung des Kontexts bei OT-Sch{\"a}tzungen, mit tiefgreifenden Implikationen f{\"u}r personalisierte therapeutische Interventionen.

Zweitens erforschen wir verschiedene \emph{dynamische} optimale Transportformulierungen, die die Verbindungen von OT zu partiellen Differentialgleichungen und Gradientenfl{\"u}ssen durch das Jordan-Kinderlehrer-Otto (JKO) Schema sowie stochastische Differentialgleichungen und optimale Kontrolle durch Diffusion Schr{\"o}dinger-Br{\"u}cken nutzen. Diese Methoden dienen als robuste Werkzeuge zur Rekonstruktion stochastischer Dynamiken aus marginalen Beobachtungen in kontinuierlicher Zeit.

Diese Dissertation integriert diese verschiedenen Konzepte unter einem einheitlichen Rahmen und bietet neue Perspektiven auf OT-Anwendungen in der Biomedizin. Die vorgestellten Methoden bilden die rechnerische und mathematische Grundlage f{\"u}r die Modellierung von Zelldynamiken und er{\"o}ffnen neue Wege zum Verst{\"a}ndnis von zellul{\"a}rer Heterogenit{\"a}t, Plastizit{\"a}t und Reaktionslandschaften. Durch die Bereitstellung neuronaler Parametrisierungen von statischem und dynamischem OT haben die entwickelten Methoden weitreichende Auswirkungen, die von der Verst{\"a}ndnis von Entwicklungspfaden bis zur Vorhersage patientenspezifischer Medikamentenreaktionen und der Gestaltung personalisierter Krebstherapien reichen.

\end{otherlanguage}

\endgroup

\vfill