%*******************************************************
% Abstract
%*******************************************************
%\renewcommand{\abstractname}{Abstract}
\pdfbookmark[1]{Abstract}{Abstract}
\begingroup
\let\clearpage\relax
\let\cleardoublepage\relax
\let\cleardoublepage\relax

\chapter*{Abstract}

Modeling dynamical systems is a core subject of many scientific disciplines as it allows us to predict future states, understand complex interactions over time, and enable informed decision-making.
Biological systems in particular are governed by dynamical processes, with their inherently complex and constantly changing patterns of interactions and behaviors.
Single-cell biology has revolutionized biomedical research, as it allows to monitor such systems at unprecedented scales.
At the same time, it presents us with formidable challenges: While single-cell high-throughput methods routinely produce millions of data points, they are destructive assays, such that the same cell cannot be observed twice nor profiled over time.
Since many of the most pressing questions in the field involve modeling the dynamic responses of heterogeneous cell populations to various stimuli, such as therapeutic drugs or developmental signals, there is a pressing need to provide computational methods that can circumvent that limitation and re-align these unpaired measurements.

Optimal transport (OT) has emerged as a major opportunity to fill in that gap \textit{in silico} as it allows us to reconstruct how a distribution evolves, given only access to \emph{distinct snapshots} of \emph{unaligned} data points.
Classical OT methods, however, do not generalize to \emph{unseen} samples. Yet, this is crucial when, for example, predicting treatment responses of incoming patient samples or extrapolating cellular dynamics beyond the measured horizon.

By harnessing the theoretical constructs of OT, this thesis explores and develops \textbf{\emph{neural}} \textbf{static} and \textbf{dynamic optimal transport} schemes for elucidating the intricate dynamics of biological populations. It encapsulates an array of algorithmic frameworks, with contributions to both the \textit{understanding} and \textit{prediction} of population dynamics:

\begin{itemize}[leftmargin=*]
	\item First, we derive \textbf{static neural optimal transport} schemes capable of learning a map between the unpaired distributions of unperturbed and perturbed cells. These models excel at predicting single-cell responses to varying perturbations, such as cancer drug screens, and generalize the inference of treatment outcomes to \emph{unobserved} cell types and patients.
This has significant implications for personalized medicine, as it allows for the prediction of treatment responses for new patients in large-scale clinical studies.

	\item Second, we explore \textbf{dynamic neural optimal transport} formulations that leverage the connections of OT to partial differential equations and gradient flows through the \acrfull{JKO} scheme, as well as stochastic differential equations and optimal control through diffusion Schr{\"o}dinger bridges. These methods then serve as robust tools for reconstructing stochastic and continuous-time dynamics from marginal observations, allowing us to dissect fine-grained and time-resolved cellular mechanisms.
\end{itemize}

This thesis connects a variety of seemingly unrelated concepts into a unified framework, and the presented methodologies offer a computational and mathematical foundation for modeling of cellular dynamics. This provides new avenues to understand cellular heterogeneity, plasticity, and response landscapes.
Such neural parameterizations of static and dynamic OT that allow for out-of-sample inference lays the groundwork for exciting opportunities to make novel biological discoveries, infer personalized  therapies from single-cell patient samples, and push the boundaries of regenerative medicine.

\endgroup

\cleardoublepage%

\begingroup
\let\clearpage\relax
\let\cleardoublepage\relax
\let\cleardoublepage\relax

\begin{otherlanguage}{ngerman}
\pdfbookmark[1]{Zusammenfassung}{Zusammenfassung}
\chapter*{Zusammenfassung}

Die Modellierung dynamischer Systeme bildet einen Schwerpunkt in vielen wissenschaftlichen Fachbereichen. Sie erm{\"o}glicht es uns, zuk{\"u}nftige Zust{\"a}nde vorherzusagen, komplexe zeitliche Interaktionen zu analysieren und fundierte Entscheidungen zu treffen. Insbesondere in biologischen Systemen, die von inh{\"a}rent komplexen und st{\"a}ndig wechselnden Interaktions- und Verhaltensmustern gesteuert werden, kommt dieser Aspekt zum Tragen.
Die Einzelzellbiologie hat dabei die biomedizinische Forschung revolutioniert, indem sie es erm{\"o}glicht, solche Systeme in beispielloser Gr{\"o}{\ss}enordnung zu messen. Gleichzeitig stellt sie uns vor gewaltige Herausforderungen: Obwohl Einzelzell-Hochdurchsatzmethoden routinem{\"a}{\ss}ig Millionen von Datenpunkten produzieren, sind es destruktive Assays, sodass dieselbe Zelle nicht wiederholt oder kontinuierlich {\"u}ber die Zeit gemessen werden kann.
Angesichts dringender Fragen zur Modellierung der dynamischen Reaktionen heterogener Zellpopulationen auf verschiedene Perturbationen, wie therapeutische Medikamente oder Entwicklungssignale, besteht ein akuter Bedarf an der Entwicklung von Algorithmen, die diese Einschr{\"a}nkung {\"u}berwinden und zeitliche Trajektorien einzelner Zellen rekonstruieren k{\"o}nnen.

Die mathematische Theorie des sogenannten optimalen Transports (OT) hat sich dabei als Schl{\"u}sselmethodik etabliert, um diese L{\"u}cke \emph{in silico} zu schlie{\ss}en. OT erlaubt es uns zu rekonstruieren, wie sich eine Verteilung {\"u}ber die Zeit hinweg entwickelt hat, selbst aus \textit{diskreten Messungen} von \textit{nicht gekoppelten Datenpunkten}. Allerdings generalisieren klassische OT Methoden nicht auf unbekannte Proben. Dies ist jedoch eine entscheidende F{\"a}higkeit f{\"u}r die Vorhersage von Behandlungsreaktionen verschiedener Patienten oder wenn zellul{\"a}re Dynamiken {\"u}ber den gemessenen Horizont hinaus extrapoliert werden sollen.

Diese Dissertation entwickelt, basierend auf den theoretischen Konzepten des optimalen Transports, sowohl \textbf{\textit{neuronale} statische} als auch \textbf{dynamische optimale Transport} Systeme, um die komplexen Dynamiken biologischer Populationen zu modellieren und zu verstehen:

\begin{itemize}[leftmargin=*]
	\item \textbf{Statischer neuronaler optimaler Transport}: 
	Diese Methoden sind in der Lage, eine Abbildung zwischen ungepaarten Verteilungen von unperturbierten und perturbierten Zellen zu erlernen. Sie erziehlen sehr gute Ergebnisse in der Vorhersage von Einzelzellreaktionen auf verschiedene Perturbationen, wie zum Beispiel Krebsmedikamenten, und erm{\"o}glichen die Vorhersage von Behandlungsreaktionen f{\"u}r neue Patienten in gro{\ss} angelegten klinischen Studien.

	\item \textbf{Dynamischer neuronaler optimaler Transport}: 
	Durch das Verkn{\"u}pfen von OT zu partiellen Differentialgleichungen und Gradientenfl{\"u}ssen durch das \acrfull{JKO} Schema, sowie stochastischer Differentialgleichungen und optimalen Steuerungen durch diffusive Schr{\"o}dinger Br{\"u}cken dienen diese Algorithmen als robuste Werkzeuge zur Rekonstruktion stochastischer und kontinuierlicher dynamischer Prozesse. Sie erm{\"o}glichen dadurch feink{\"o}rnige Analysen zeitlich aufgel{\"o}ster zellul{\"a}rer Mechanismen.
\end{itemize}

Diese Dissertation verbindet eine Vielzahl scheinbar nicht verwandter Konzepte in einem einheitlichen Rahmen, bietet eine methodische und mathematische Grundlage f{\"u}r die Modellierung zellul{\"a}rer Dynamiken und erlaubt damit, zellul{\"a}re Heterogenit{\"a}t, Plastizit{\"a}t und Reaktionen zu Perturbationen besser zu verstehen. Die entwickelten neuronalen Parameterisierungen von statischen und dynamischen OT, die Inferenzen au{\ss}erhalb der Stichprobe zulassen, er{\"o}ffnen einen aufregenden Forschungsweg f{\"u}r die Zukunft. Sie bieten M{\"o}glichkeiten f{\"u}r neuartige biologische Entdeckungen und die Vorhersage personalisierter Therapien aus Einzelzell-Patientenproben und erweitern die Grenzen der regenerativen Medizin.


\end{otherlanguage}

\endgroup

\vfill